\chapter{einleitungBig}
In der sich ständig weiterentwickelnden Domäne der Elektronik und Elektrotechnik ist es zunehmend evident, dass bestimmte Schlüsselkomponenten, insbesondere Kapazitäten und Induktivitäten, eine Neigung zur Degradation aufgrund vielfältiger Umwelteinflüsse aufweisen. Diese Degradationsprozesse können signifikante Auswirkungen auf die Funktionalität und Integrität der betroffenen Schaltungen haben. Insbesondere könnte man postulieren, dass die Lebensdauer und Effizienz von elektronischen Systemen erheblich beeinträchtigt werden könnten, wenn diese Degradationsphänomene nicht sorgfältig berücksichtigt und adressiert werden.

Zur Untermauerung dieses Gedankens gibt es eine Fülle von wissenschaftlichen Untersuchungen, die die unterschiedlichen Degradationsmechanismen und ihre Auswirkungen beleuchten. Beispielsweise haben Jeong, Jaeyoon, Kwak, Sangshin und Choi, Seungdeog in ihrem Artikel "Degradation-Sensitive Control Algorithm Based on Phase Optimization for Interleaved DC–DC Converters" spezifische Degradationsprozesse in DC-DC-Wandlern aufgezeigt, wobei insbesondere die Änderung des äquivalenten Serienwiderstands (ESR) von Kondensatoren als ein Hauptindikator für solche Degradationsphänomene hervorgehoben wurde\cite[p.~1]{jeong2023degradation}.

In einem ähnlichen Kontext haben Kulkarni, Chetan, Biswas, Gautam, Kim, Kyusung und Bharadwaj, Raj Mohan die systemischen Auswirkungen von Degradationen auf kritische Avioniksysteme hervorgehoben. Ihre Studien legen nahe, dass solche Degradationen tiefgreifende Auswirkungen auf Navigationssysteme wie GPS- und Inertial-Navigationssysteme haben können\cite[p.~3]{kulkarni_model-based_2023}.

Diese Erkenntnisse unterstreichen die Notwendigkeit, die Degradationsmechanismen elektronischer Komponenten sowohl auf Makro- als auch auf Mikroebene eingehend zu verstehen, um innovative Lösungen zu entwickeln, die solche Phänomene mitigieren oder gar verhindern können.
