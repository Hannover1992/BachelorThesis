\chapter{Grundlagen}

Grundlagenkapitel:

Im Grundlagenkapitel erklären Sie die einzelnen Techniken, Methoden oder Konzepte, die in Ihrem Experiment verwendet werden, detailliert. Dies hilft den Lesern, ein grundlegendes Verständnis für die verwendeten Techniken zu entwickeln und die Kontextualisierung des Experiments zu verstehen.
Durch das Herunterbrechen und Erklären der einzelnen Techniken bereiten Sie die Leser darauf vor, die Komplexität Ihres Experiments zu verstehen.

Basierend auf den Informationen in dem Text können die folgenden Themen für die Literaturrecherche im Kapitel "Grundlagen" der Arbeit erwähnt werden:


1. Gleichspannungswandler (DC-DC-Konverter)


2. Alterungsbedingte Degradation von Schaltungskomponenten

3. Neuronale Netze
4. Aktivierungs FUnktinen
5. Gredien Descent
6. Phenomen Des Deep Learnign
7. Deep Q Learing
8. DDPG
4. Trainingsprozess und Architektur neuronaler Netze
3. Verwendung neuronaler Netze zur Schaltungsoptimierung
5. Anwendung von Deep Deterministic Policy Gradient (DDPG) und Bayesscher Optimierung in der Schaltungsoptimierung
6. Transientenanalyse mit SystemC zur Bewertung von Simulationsergebnissen
7. Hyperparameteroptimierung für neuronale Netze
8. Vergleich von neuronalen Netzwerk-Controllern mit traditionellen PID- und digitalen Gleitmodus-Reglern
9. Anwendung von neuronalen Netzen zur Überwachung und Anpassung von Schaltungsdegradation
10. Verifizierung der Schaltungsoptimierung durch Triangulation
11. Herausforderungen und Lösungsansätze bei der neuronalen Netzarchitektur und dem Training
12. Ergebnisse und Implikationen für zukünftige Forschungen

Dies sind nur Vorschläge und die genaue Struktur und Auswahl der Themen hängt von den spezifischen Zielen der Arbeit ab.


\subsection{Grundlagen des Buck-Konverters in DC-DC-Wandlern}

Die Wandlung von Gleichspannung (DC) in eine andere Gleichspannung ist ein kritischer Aspekt in der Elektronik und Energieversorgung. Ein weit verbreitetes Schaltungsdesign, das diese Funktion ausführt, ist der Buck-Konverter. In der Literatur wird dieser als eine Standardmethode für DC-DC-Wandlung beschrieben \cite{Luo2004}.

\subsubsection{Hauptkomponenten und Funktionen}

\begin{itemize}
    \item \textbf{MOSFET-Transistor}: Der MOSFET-Transistor dient als elektronischer Schalter, der den Stromfluss durch die Schaltung steuert. Wie erwähnt, kann der Einsatz von MOSFET-Transistoren die Effizienz signifikant erhöhen, da sie niedrigere Verluste im Vergleich zu anderen Schaltern haben.
    \item \textbf{Induktivität (Spule)}: Diese speichert temporär Energie in Form eines Magnetfelds, wenn Strom hindurchfließt.
    \item \textbf{Diode}: Die Diode erlaubt den Stromfluss nur in einer Richtung, insbesondere wenn der MOSFET ausgeschaltet ist.
    \item \textbf{Kondensator}: Der Kondensator glättet die Ausgangsspannung und speichert Energie für die Last.
\end{itemize}

\subsubsection{Regelung und Anwendungen}

In der Praxis werden Buck-Konverter oft von einer nicht-idealen Spannungsquelle gespeist und müssen daher unter variablen Eingangsspannungen und Lastströmen arbeiten \cite{choi2013pulsewidth}. Daher ist eine geschlossene Regelungsschleife erforderlich, um eine konstante Ausgangsspannung sicherzustellen.

Buck-Konverter finden eine breite Anwendung in verschiedenen elektronischen Geräten und Systemen. Ihr hoher Wirkungsgrad, der in der Regel zwischen 75\% und 98\% liegt, macht sie besonders attraktiv.




