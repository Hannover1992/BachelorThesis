\chapter{Grundlagen}

Grundlagenkapitel:

Im Grundlagenkapitel erklären Sie die einzelnen Techniken, Methoden oder Konzepte, die in Ihrem Experiment verwendet werden, detailliert. Dies hilft den Lesern, ein grundlegendes Verständnis für die verwendeten Techniken zu entwickeln und die Kontextualisierung des Experiments zu verstehen.
Durch das Herunterbrechen und Erklären der einzelnen Techniken bereiten Sie die Leser darauf vor, die Komplexität Ihres Experiments zu verstehen.

Basierend auf den Informationen in dem Text können die folgenden Themen für die Literaturrecherche im Kapitel "Grundlagen" der Arbeit erwähnt werden:


1. Gleichspannungswandler (DC-DC-Konverter)


2. Alterungsbedingte Degradation von Schaltungskomponenten

3. Neuronale Netze
4. Aktivierungs FUnktinen
5. Gredien Descent
6. Phenomen Des Deep Learnign
7. Deep Q Learing
8. DDPG
4. Trainingsprozess und Architektur neuronaler Netze
3. Verwendung neuronaler Netze zur Schaltungsoptimierung
5. Anwendung von Deep Deterministic Policy Gradient (DDPG) und Bayesscher Optimierung in der Schaltungsoptimierung
6. Transientenanalyse mit SystemC zur Bewertung von Simulationsergebnissen
7. Hyperparameteroptimierung für neuronale Netze
8. Vergleich von neuronalen Netzwerk-Controllern mit traditionellen PID- und digitalen Gleitmodus-Reglern
9. Anwendung von neuronalen Netzen zur Überwachung und Anpassung von Schaltungsdegradation
10. Verifizierung der Schaltungsoptimierung durch Triangulation
11. Herausforderungen und Lösungsansätze bei der neuronalen Netzarchitektur und dem Training
12. Ergebnisse und Implikationen für zukünftige Forschungen

Dies sind nur Vorschläge und die genaue Struktur und Auswahl der Themen hängt von den spezifischen Zielen der Arbeit ab.


\input{21DCDC}
