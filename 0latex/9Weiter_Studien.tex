\chapter{Zukünftige Arbeit}
\label{sec:Future_Work}

Die Ergebnisse dieser Studie eröffnen verschiedene Wege für zukünftige Forschungen im Bereich der Regelungstechnik und des maschinellen Lernens. Die Modularität und Generalisierbarkeit des verwendeten Ansatzes bieten zahlreiche Möglichkeiten für weiterführende Untersuchungen.

\paragraph{Anwendung auf Verschiedene Schaltungen}
Ein vielversprechender Bereich für zukünftige Arbeiten ist die Anwendung des entwickelten Reinforcement Learning-Algorithmus auf eine Vielzahl von Schaltungen. Die Flexibilität des Systems ermöglicht es, mit unterschiedlichen Schaltungsdesigns und Belohnungsfunktionen zu experimentieren, um ihre Wirksamkeit in verschiedenen Kontexten zu testen.

\paragraph{Optimierung der Netzwerkarchitektur}
Die Ergebnisse hinsichtlich der Miniaturisierung der Netzwerke legen nahe, dass eine weitere Optimierung der Netzwerkarchitektur möglich ist. Insbesondere könnte die Kombination eines komplexeren Kritikers mit einem minimalisierten Akteur untersucht werden. Zusätzlich wäre die Anwendung von spezialisierten Netzwerkoptimierungsalgorithmen, die auf die Reduktion der Netzwerkgröße abzielen, ein interessantes Forschungsfeld.

\paragraph{Validierung in der Realen Welt}
Ein kritischer Schritt für die Zukunft ist die Überprüfung der Simulationsergebnisse in realen Anwendungen. Es ist unerlässlich, die in der simulierten Umgebung erzielten Ergebnisse in der realen Welt zu validieren und dabei zusätzliche Variablen wie Rauschen und unvorhersehbare Betriebsbedingungen einzubeziehen. Dies würde nicht nur die Zuverlässigkeit des Algorithmus bestätigen, sondern auch seine Anwendbarkeit in praktischen Szenarien.

\paragraph{Erweiterte Reinforcement Learning Strategien}
Es wäre lohnenswert, alternative Reinforcement Learning-Strategien und -Architekturen zu erforschen, um die Effektivität und Effizienz des Lernprozesses weiter zu verbessern. Insbesondere könnten Anpassungen der Lernrate oder die Integration von Techniken zur Vermeidung von Overfitting den Algorithmus robuster und vielseitiger machen.

\paragraph{Parallele Architekturen und Simulationen}
Eine mögliche Erweiterung des aktuellen Ansatzes wäre die Nutzung paralleler Architekturen mit mehreren Kritikern und Akteuren. Dies könnte die Effizienz und Performance in komplexen Systemen steigern, insbesondere durch die Reduktion des benötigten Speichers und der Rechenzeit. Solche parallelen Strukturen könnten insbesondere bei größeren Schaltungen von Vorteil sein, um den Lernprozess zu beschleunigen.

\paragraph{Anpassung der Belohnungsfunktion}
Die Bedeutung einer maßgeschneiderten Belohnungsfunktion ist im Rahmen dieser Studie deutlich hervorgetreten. Zukünftige Forschungen könnten sich darauf konzentrieren, die Belohnungsfunktion speziell auf die erwartete Nutzung des Systems abzustimmen, wie z.B. die Anpassung an bestimmte Verhaltensmuster oder Betriebsbedingungen.

\paragraph{Optimales Schaltungsentwurf}
Ein weiterer Ansatz könnte darin bestehen, die Architektur des Algorithmus zur Entwicklung optimaler Schaltungsentwürfe zu nutzen. Anstatt sich lediglich auf die Regelung bestehender Schaltungen zu konzentrieren, könnte der Algorithmus dazu verwendet werden, Parameter für den Entwurf effizienterer und leistungsfähigerer Schaltungen zu ermitteln.

\paragraph{Erweiterung der Anwendbarkeit}
Die Verallgemeinerungsfähigkeit des Ansatzes legt nahe, dass er auch auf andere Arten von Schaltungen und Regelungsproblemen anwendbar sein könnte. Die Möglichkeit, den Algorithmus für eine breite Palette von Anwendungen zu nutzen, eröffnet ein weites Feld für zukünftige Experimente und Entwicklungen.

Insgesamt bieten die Ergebnisse dieser Arbeit eine solide Basis für eine Vielzahl von zukünftigen Forschungsprojekten, die darauf abzielen, die Leistungsfähigkeit und Effizienz von Regelungssystemen durch maschinelles Lernen weiter zu verbessern.
