\begin{abstract}
In dieser Arbeit wird die Verwendung neuronaler Netze zur Optimierung und Steuerung eines PID-regulierten DC-Konverters untersucht. Das Ziel besteht darin, ein System zu entwickeln, das in der Lage ist, sowohl die altersbedingte Degradation als auch Fertigungstoleranzen von Schaltungskomponenten wie Kapazität und Induktivität zu überwachen und anzupassen, um eine dauerhaft optimale Leistung des Konverters sicherzustellen. Ein besonderer Fokus liegt auf dem Trainingsprozess und der Architektur des neuronalen Netzes. Der Trainingsprozess wird mithilfe von Methoden wie Deep Deterministic Policy Gradient (DDPG) und Bayesscher Optimierung umgesetzt. Das Training und die Schaltungssimulation werden unter Einsatz von Transientenanalyse mit SystemC durchgeführt, um eine präzise Bewertung und Auswertung der Simulationsergebnisse zu ermöglichen. Es werden Techniken zur Optimierung der Hyperparameter des neuronalen Netzes vorgestellt. Herausforderungen und Lösungsansätze im Kontext der neuronalen Netzarchitektur und des Trainings werden diskutiert. Abschließend werden die erzielten Ergebnisse und ihre Implikationen für zukünftige Forschungen präsentiert.
\end{abstract}

