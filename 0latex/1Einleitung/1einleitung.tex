%\chapter{Einleitung}
%@START In der modernen Elektrotechnik sind DC-Konverter essentiell, um Energie effizient und präzise von einer Quelle zu einem Verbraucher zu leiten.@END [Grundlagen der Elektrotechnik]
%
%@START Mit der Zeit können jedoch Schaltungskomponenten wie Kapazitäten und Induktivitäten durch verschiedene Umwelteinflüsse degradieren, was ihre Funktionsweise beeinträchtigt.@END [Studien zur Degradation von Elektronikkomponenten]
%
%@START Hier stellt sich die Frage: Wie kann man die Leistungsfähigkeit dieser Konverter trotz solcher Alterungsprozesse aufrecht erhalten?@END [Herausforderungen bei der Wartung von Elektronikkomponenten]
%
%@START Die Antwort könnte in der Anwendung von künstlichen neuronalen Netzen liegen.@END [Einführung in künstliche neuronale Netze]
%
%
%@START In dieser Arbeit wird untersucht, wie neuronale Netze dazu genutzt werden können, um die Degradation von Schaltungskomponenten zu überwachen und die PD-Koeffizienten eines DC-Konverters entsprechend anzupassen. Dabei werden Themen wie die Architektur des Netzes, Trainingsmethoden und -umgebungen, sowie die Herausforderungen beim Training und deren Lösungsansätze behandelt.@END [Spezifische Studien zur Optimierung von DC-Konvertern mit neuronalen Netzen]


\input{1Einleitung/11einleitungBIG}
