\chapter{Einleitung} 
Die effiziente und präzise Lenkung von Energie von einer Quelle zu einem Verbraucher stellt eine zentrale Herausforderung in der modernen Elektrotechnik dar. Gleichspannungswandler (DC-DC Konverter) spielen hier eine entscheidende Rolle \cite[p.~70]{wensdesign2022}. Es existieren verschiedene Methoden zur DC-DC-Spannungsumwandlung, jede mit ihren spezifischen Vor- und Nachteilen, abhängig von unterschiedlichen Betriebsbedingungen und Spezifikationen \cite[p.~70]{wensdesign2022}.

Es wird zunehmend klar, dass bestimmte Schlüsselkomponenten, insbesondere Kapazitäten, eine Tendenz zur Degradation aufweisen. Diese Degradation ist häufig auf vielfältige Umwelteinflüsse zurückzuführen. Sie kann signifikante Auswirkungen auf die Funktionalität und Integrität der betroffenen Schaltungen haben. Daher ist es plausibel anzunehmen, dass die Lebensdauer und Effizienz von elektronischen Systemen erheblich beeinträchtigt werden könnten, wenn diese Degradationsmechanismen nicht sorgfältig betrachtet und adressiert werden.

Wissenschaftliche Untersuchungen stützen diese Beobachtungen. Jeong et al. haben in ihrem Artikel "Degradation-Sensitive Control Algorithm Based on Phase Optimization for Interleaved DC–DC Converters" spezifische Degradationsprozesse in DC-DC-Wandlern aufgezeigt. Dabei wurde insbesondere der äquivalente Serienwiderstand (ESR) von Kondensatoren als ein Hauptindikator für Degradation identifiziert \cite[p.~1]{jeong2023degradation}.

In einem ähnlichen Kontext haben Kulkarni et al. die systemischen Auswirkungen von Degradationen auf kritische Avioniksysteme hervorgehoben. Ihre Studien zeigen, dass solche Degradationen ernsthafte Konsequenzen für Navigationssysteme wie das Global Positioning System (GPS) und Inertial-Navigationssysteme haben können \cite[p.~3]{kulkarni_model-based_2023}.

Diese Erkenntnisse betonen die Notwendigkeit, die Mechanismen der Degradation elektronischer Komponenten auf Makro- und Mikroebene genau zu verstehen. Nur so können innovative Lösungen entwickelt werden, die solche Phänomene minimieren oder sogar verhindern können.

Ein vielversprechender Ansatz könnte in der Anwendung von künstlichen neuronalen Netzen (KNN) liegen. Wie Steven L. Brunton und J. Nathan Kutz in ihrem Werk "Data-Driven Science and Engineering: Machine Learning, Dynamical Systems, and Control" darlegen, bieten KNN ausgezeichnete Möglichkeiten zur Steuerung komplexer, nichtlinearer Systeme, einschließlich elektronischer Schaltungen \cite[p.~270]{brunton2019data}. Almawlawe et al. konnten in ihrer Studie zeigen, dass neuronale Netzwerk-Controller im Vergleich zu traditionellen Proportional-Integral-Derivative (PID)- und digitalen Gleitmodus-Reglern eine überlegene Leistung bei der Ausgangsspannungsverfolgung eines Buck DC/DC-Konverters bieten \cite[p.~8]{Almawlawe2023}. Miguel Morales betont in "Grokking Deep Reinforcement Learning", dass KNN einer der leistungsfähigsten Funktionsapproximatoren sind und oft andere Methoden übertreffen \cite[p.~22]{morales2020grokking}.

In dieser Arbeit wird untersucht, wie KNN dazu genutzt werden können, um die Degradation von Schaltungskomponenten zu überwachen und die PID-Koeffizienten eines DC-Konverters entsprechend anzupassen. Themen wie die Architektur des Netzes, Trainingsmethoden und -umgebungen, sowie Herausforderungen beim Training und deren Lösungsansätze werden behandelt.
