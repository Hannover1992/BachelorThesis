
\subsection{Bedeutung der Lernrate}
\label{sec: Learnign Rate}

Die Lernrate ist ein kritischer Hyperparameter, der die Geschwindigkeit der Modellanpassung steuert. Wie aus der Literatur bekannt ist, kann eine falsch gewählte Lernrate den Lernprozess verlangsamen oder sogar zu einer fehlgeschlagenen Konvergenz führen \cite{russell2021ai}. In ähnlicher Weise betont Heaton, dass die Wahl der Lernrate und des Momentums entscheidend für die Leistung des Trainings ist \cite{heaton_2012}.

\paragraph{subsectionHerausforderungen und Lösungsansätze}

Die Einstellung der Lernrate ist häufig ein Prozess des Ausprobierens, und die beste Methode zur Ermittlung der optimalen Lernrate ist nicht eindeutig. Goodfellow et al. erklären, dass die Lernrate das effektive Kapazitätsmodell in einer komplexeren Weise steuert als andere Hyperparameter \cite{Goodfellow-et-al-2016}. Daher ist es von entscheidender Bedeutung, sowohl das Training als auch den Testfehler zu überwachen, um zu diagnostizieren, ob das Modell unter- oder überangepasst ist.
