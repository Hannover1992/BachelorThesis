\subsection{Einführung in Neuronale Netzwerke}
Neuronale Netzwerke bilden das rechnerische Grundgerüst für eine Vielzahl von Aufgaben in den Berei-chen maschinelles Lernen und künstliche Intelligenz. Sie sind von dem komplexen Netzwerk an Neuronen im menschlichen Gehirn inspiriert und versuchen, biologische Lernprozesse nachzuahmen \cite{aggarwal_neural_networks_2018}. In diesem Zusammenhang bieten sie ein robustes und flexibles Rahmenwerk zur Lösung komplexer Herausforderungen \cite{Goodfellow-et-al-2016}.

\subsubsection{Vorteile von Neuronalen Netzwerken}
Neuronale Netzwerke bieten mehrere entscheidende Vorteile, die ihren Einsatz in unterschiedlichen Anwendungsbereichen attraktiv machen:
\begin{itemize}
    \item \textbf{Parallelität:} Sie sind für die parallele Verarbeitung konzipiert und ermöglichen daher schnelle Berechnungen sowie Echtzeitverarbeitung.
    \item \textbf{Nichtlineare Funktionsapproximation:} Die Netzwerke sind besonders gut geeignet, nichtlineare Funktionen zu approximieren \cite{Goodfellow-et-al-2016}, was sie vielseitig einsetzbar macht.
    \item \textbf{Modellgeneralisierung:} Neuronale Netzwerke können aus einer begrenzten Datenmenge gene-ralisieren und somit präzise Vorhersagen für unbekannte Eingaben treffen.
\end{itemize}
Diese Vorteile bilden die Grundlage für ihre breite Anwendbarkeit, die im nächsten Abschnitt erläutert wird.

\subsubsection{Unterschied zwischen Biologischen und Künstlichen Neuronalen Netzwerken}
Künstliche neuronale Netzwerke bestehen aus rechnerischen Einheiten, den sogenannten Neuronen. Diese sind durch anpassbare Gewichtungen verbunden, die der Stärke synaptischer Verbindungen in biologischen Systemen ähneln. Lernen erfolgt durch die Anpassung dieser Gewichtungen, ähnlich wie sich die Stärken synaptischer Verbindungen in biologischen Systemen als Reaktion auf Reize ändern \cite{aggarwal_neural_networks_2018}.

\subsubsection{Deep Learning als Spezialisierung}
Deep Learning stellt einen spezialisierten Unterbereich des maschinellen Lernens dar, der neuronale Netzwerke mit drei oder mehr Schichten verwendet. Diese tiefen Netzwerke führen eine hierarchische Merkmalsextraktion durch, die es ihnen ermöglicht, immer komplexere Muster und Merkmale zu erkennen, während die Daten durch die Schichten fließen.

\subsubsection{Zusammenfassung und Ausblick}
Zusammenfassend bieten neuronale Netzwerke ein leistungsfähiges Rahmenwerk für eine Vielzahl von Aufgaben, von einfacher Mustererkennung bis hin zu komplexen Entscheidungsfindungsprozessen. Der Einsatz von mehrschichtigen Architekturen und nichtlinearen Aktivierungsfunktionen erweitert die Fähig-keiten traditioneller maschineller Lernalgorithmen \cite{aggarwal_neural_networks_2018}.

Mit dieser Grundlage werden die folgenden Abschnitte einen vertieften Einblick in die mathematischen Aspekte von neuronalen Netzwerken bieten. Insbesondere werden wir uns darauf konzentrieren, wie neuronale Netzwerke bei der Optimierung und Steuerung von PID-regulierten DC-Konvertern Anwendung finden. Dabei liegt der Fokus auf der Berücksichtigung von Alterungsprozessen und Abnutzung von Schaltungselementen wie Kapazitäten und Induktivitäten und wie diese Einflüsse mathematisch modelliert und optimiert werden können.
