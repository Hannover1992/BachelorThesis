\chapter{Grundlagen}




\textbf{Einleitung zum Kapitel}

Dieses Kapitel dient als umfassende Grundlage für die Erforschung der Rolle neuronaler Netze in der Optimierung und Steuerung von PID-regulierten DC-Konvertern. Im Fokus stehen sowohl die Grundlagen der DC-DC-Konvertertechnologie als auch spezielle Herausforderungen, die in diesem Kontext auftreten können, wie beispielsweise die altersbedingte Degradation von Schaltungskomponenten. Darüber hinaus bietet das Kapitel einen Überblick über moderne Optimierungsmethoden wie DDPG (Deep Deterministic Policy Gradients) und Bayessche Optimierung, die in der aktuellen Forschung Bedeutung erlangt haben.

Der Inhalt dieses Kapitels zielt darauf ab, den Leser umfassend auf die Herausforderungen, technischen Lösungen und innovativen Ansätze in diesem sich schnell entwickelnden Forschungsfeld vorzubereiten.

%1. Gleichspannungswandler (DC-DC-Konverter)


\section{Elektrotechnik}
\subsection{Grundlagen des Buck-Konverters in DC-DC-Wandlern}

Die Wandlung von Gleichspannung (DC) in eine andere Gleichspannung ist ein kritischer Aspekt in der Elektronik und Energieversorgung. Ein weit verbreitetes Schaltungsdesign, das diese Funktion ausführt, ist der Buck-Konverter. In der Literatur wird dieser als eine Standardmethode für DC-DC-Wandlung beschrieben \cite[p.~66]{wensdesign2022}.

\subsubsection{Hauptkomponenten und Funktionen}

\subsection{Hauptkomponenten und Funktionen eines DC-DC-Konverters}

\subsubsection{MOSFET-Transistor}
Der MOSFET-Transistor agiert als elektronischer Schalter, der den Stromfluss in der Schaltung reguliert. Im Vergleich zu alternativen Schaltelementen bietet der MOSFET eine signifikante Effizienzsteigerung durch minimale Leistungsverluste. Dies wird durch Phänomene wie Trägermobilität und die damit verbundene Widerstandsfähigkeit gegenüber thermischen Ausfällen ermöglicht \cite[p.~29]{choi2013pulsewidth}.

\subsubsection{Induktivität (Spule)}
Die Induktivität dient der temporären Energiespeicherung in Form eines magnetischen Feldes, das beim Stromfluss durch die Spule generiert wird. Dies ist insbesondere relevant in Anwendungen wie Solenoid-Antriebsschaltungen, wo die Induktivität als Energiespeicher und -überträger fungiert \cite[p.~54]{choi2013pulsewidth}.

\subsubsection{Diode}
Die Diode ist so ausgerichtet, dass sie den Strom nur in einer Richtung passieren lässt. Dies ist insbesondere wichtig, wenn der MOSFET-Transistor deaktiviert ist. Als passive Schalter werden oftmals schnelle Erholungsdioden oder Schottky-Dioden aufgrund ihrer exzellenten Schalteigenschaften verwendet \cite[p.~29]{choi2013pulsewidth}.

\subsubsection{Kondensator}
Der Kondensator dient der Glättung der Ausgangsspannung und speichert Energie für die Last. Er spielt eine wichtige Rolle in der Dynamik der Schaltung und ermöglicht eine stabilere Energieversorgung \cite[p.~54]{Kularatna2012}.
\subsubsection{Regelung und Anwendungen}

In der Praxis werden Buck-Konverter oft von einer nicht-idealen Spannungsquelle gespeist und müssen daher unter variablen Eingangsspannungen und Lastströmen arbeiten \cite[p.~124,120,113]{choi2013pulsewidth}.. Daher ist eine geschlossene Regelungsschleife erforderlich, um eine konstante Ausgangsspannung sicherzustellen.

Buck-Konverter finden eine breite Anwendung in verschiedenen elektronischen Geräten und Systemen. Ihr hoher Wirkungsgrad, der in der Regel zwischen 75\% und 98\% liegt, macht sie besonders attraktiv.


\begin{figure}[htbp]
    \centering
    \includegraphics[width=0.7\linewidth]{2Grundlagen/111DCDC.png}
    \caption{Schematische Darstellung eines DC-DC Konverters. Quelle: \cite[Seite 88]{choi2013pulsewidth}}
    \label{fig:dcdc_converter}
\end{figure}




\input{2Grundlagen/12Degradation}
\subsection{PID-Regler}
Der PID-Regler (Proportional-Integral-Derivativ) ist eine weit verbreitete Regelungsstrategie in industriellen Steuerungssystemen und verschiedenen Arten von Anwendungen. Er ist unerlässlich für die Regelung von Prozessen wie Geschwindigkeit, Temperatur und Spannung~\cite[p.~2]{Hussein2011PIDGA}.

\paragraph{Proportionalanteil (P)}
Diese Komponente erzeugt einen Ausgangswert, der proportional zum aktuellen Fehlerwert ist. Die proportionale Reaktion kann durch Multiplikation des Fehlers mit einer Konstanten namens \( K_p \) eingestellt werden, die als Proportionalverstärkung bezeichnet wird.
\begin{equation}
P_{\text{out}} = K_p \times e(t)
\end{equation}

\paragraph{Integralanteil (I)}
Diese Komponente befasst sich mit der Akkumulation vergangener Fehler. Wenn der Fehler über einen längeren Zeitraum vorhanden war, wird er akkumuliert (Integral des Fehlers), und der Regler wird den Steuerausgang in Beziehung zu einer Konstanten \( K_i \) ändern, die als Integralverstärkung bekannt ist.
\begin{equation}
I_{\text{out}} = K_i \times \int e(t) \, dt
\end{equation}

\paragraph{Differentiantanteil (D)}
Diese Komponente liefert einen Steuerausgang, um die Änderungsrate des Fehlers zu kompensieren. Der Beitrag des Differenzierungsanteils zur gesamten Steueraktion wird als Differenzierungsverstärkung \( K_d \) bezeichnet.
\begin{equation}
D_{\text{out}} = K_d \times \frac{d}{dt} e(t)
\end{equation}
\cite[p.~1744]{russell2021ai}

\paragraph{Die PID-Regelungsgleichung}
Die PID-Regelungsgleichung kombiniert diese drei Komponenten, um den Steuerausgang zu erzeugen:
\begin{equation}
\text{Steuerausgang} = P_{\text{out}} + I_{\text{out}} + D_{\text{out}}
\end{equation}
\begin{equation}
\text{Steuerausgang} = (K_p \times e(t)) + (K_i \times \int e(t) \, dt) + (K_d \times \frac{d}{dt} e(t))
\end{equation}

\paragraph{Einstellung der Verstärkungsfaktoren}
Die Konstanten \( K_p, K_i, \) und \( K_d \) werden eingestellt, um die optimale Systemleistung zu erreichen; ein schlecht eingestellter PID-Regler kann instabil, langsam oder schwingend sein.

\paragraph{Anwendungen bei Gleichstrom-Gleichstrom-Wandlern}
Im Kontext von Gleichstrom-Gleichstrom-Wandlern können PID-Regler helfen, die Ausgangsspannung zu stabilisieren, indem sie die Ausgangsspannung kontinuierlich mit der gewünschten Spannung vergleichen und handeln, um den Fehler durch Anpassung des Tastverhältnisses des Schaltelements zu minimieren~\cite[p.~4]{Almawlawe2023}.

\paragraph{Fazit}
Der PID-Regler ist eine vielseitige und weit verbreitete Regelungsstrategie. Seine Anpassungsfähigkeit und Effizienz machen ihn ideal für eine breite Palette von Anwendungen, von industriellen Prozessen bis zu modernen Technologiesystemen. Für eine erweiterte Diskussion über verschiedene Varianten von PID-Reglern, wie zum Beispiel den Fuzzy PID-Controller, könnten Sie das Paper "Shi2020AdaptiveController" verwenden~\cite[p.~9]{Shi2020AdaptiveController}.


\begin{figure}[htbp]
    \centering
    \includegraphics[width=0.6\linewidth]{2Grundlagen/13PID.png}
    \caption{Schematische Darstellung eines PID-Regler. Quelle: \cite[p.~18]{SwainBaid2014}}
    \label{fig:PID_converter}
\end{figure}



\subsection{Pulsweitenmodulation und ihre Darstellung}
Pulsweitenmodulation (PWM) ist eine Schlüsseltechnik in DC-DC-Wandlern, die zur Steuerung der Schaltkomponenten eingesetzt wird, um die Ausgangsspannung oder den Ausgangsstrom zu regulieren. Sie ermöglicht eine präzise Kontrolle, indem sie die 'Einschaltzeit' des Schalters im Vergleich zur gesamten Zykluszeit (Einschaltzeit + Ausschaltzeit) variiert.\cite[p.~2]{peddapelli2017pulse}

\paragraph{Tastverhältnis \(D\)}
Das Tastverhältnis \( D \) wird mathematisch als das Verhältnis der Einschaltzeit zur gesamten Zykluszeit beschrieben:
\begin{equation}
D = \frac{\text{Einschaltzeit}}{\text{Einschaltzeit} + \text{Ausschaltzeit}}
\end{equation}

\paragraph{Proportionalanteil (P)}
Das Tastverhältnis spielt eine wichtige Rolle, da es den Mittelwert der Ausgangsspannung oder des Ausgangsstroms bestimmt. Bei der PWM wird ein Steuersignal mit einem hochfrequenten Trägersignal verglichen, um die 'Ein'- und 'Aus'-Zustände des Schalters festzulegen. Das Steuersignal stammt oft von höheren Regelkreisen wie PID-Reglern, die den Fehler zwischen Soll- und Istwert minimieren sollen.

Die Hauptmotivation für die Verwendung von PWM in Steuerungssystemen ist die Anpassung des Mittelwerts der Ausgabe an ein Referenzsignal. Zusätzlich wird versucht, harmonische Verzerrungen und Schaltverluste zu minimieren \cite[p.~2]{peddapelli2017pulse}.

In der Abbildung oben ist eine typische PWM-Schaltung dargestellt. Die PWM-Block und die Spannungsrückführungsschaltung im DC-DC-Wandler arbeiten zusammen, um sicherzustellen, dass die Ausgangsspannung der Referenzspannung im stationären Zustand folgt. Hierbei wird ein Steuersignal \(v_{\text{con}}\) und ein Rampensignal \(V_{\text{ramp}}\) verwendet, um die Impulsbreite des aktiven Schalters zu modulieren. Das rechte Diagramm (b) zeigt die Steuersignale und ihre Relation zueinander, wodurch das Schaltverhalten des Wandlers beeinflusst wird \cite[p.~114]{choi2013pulsewidth}.



\begin{figure}[htbp]
    \centering
    \includegraphics[width=0.8\linewidth]{2Grundlagen/141PWM.png}
    \caption{Schematische Darstellung eines PWM-Modulator. Quelle: \cite[p.~114]{choi2013pulsewidth}}
    \label{fig:PWM_converter}
\end{figure}


\section{Informationstechnologie}

\input{2Grundlagen/21NeuronaleNetze}
\subsection{Vorwärtspropagation in Neuronalen Netzwerken}
\subsubsection{Schicht-für-Schicht-Propagation}
Beginnend mit der Eingabeschicht \( A^{[0]} \), die im Wesentlichen die Eingabedaten \( X \) sind, berechnet jede nachfolgende Schicht \( Z^{[l]} \) und \( A^{[l]} \) entsprechend den oben genannten Gleichungen. Dies bildet das Kernstück der Vorwärtspropagation.

\subsubsection{Dimensionalität und Netzwerkarchitektur}
Die Anzahl der Neuronen in jeder Schicht und die Art der verwendeten Aktivierungsfunktion können die Leistung des Netzwerks erheblich beeinflussen. Es ist wichtig, die Dimensionalität jeder Schicht während der Entwurfsphase zu berücksichtigen, um ein effektives Lernen sicherzustellen.

Die Vorwärtspropagation ist ein wesentlicher Prozess in neuronalen Netzwerken, der die Übertragung von Eingabedaten durch die Netzwerkarchitektur ermöglicht, um die Ausgabe zu erzeugen \cite[p.~1421]{russell2021ai}. Sie ist eine Abfolge von mathematischen Operationen, die Gewichtungen, Biases und Aktivierungsfunktionen involvieren \cite[p.~73]{Chollet2021}.

\subsubsection{Gewichtsmatrix \( W^{[l]} \) und Bias-Vektor \( b^{[l]} \)}
Die Gewichtsmatrix für die Schicht \( l \) wird als \( W^{[l]} \) bezeichnet, und \( b^{[l]} \) ist der Bias-Vektor für dieselbe Schicht \cite[p.~46]{heaton_2012}. Diese Parameter werden während des Backpropagation-Prozesses trainiert, um den Fehler zwischen der vorhergesagten und der tatsächlichen Ausgabe zu minimieren \cite[p.~41]{aggarwal_neural_networks_2018}.

\begin{equation}
Z^{[l]} = W^{[l]} A^{[l-1]} + b^{[l]}
\end{equation}

\subsubsection{Aktivierungsfunktionen}
Eine Aktivierungsfunktion, normalerweise durch \( \sigma \) bezeichnet, transformiert die gewichtete Summe \( Z^{[l]} \) in die aktivierte Ausgabe \( A^{[l]} \) \cite[p.~1421]{russell2021ai}.

\begin{equation}
A^{[l]} = \sigma(Z^{[l]})
\end{equation}

\begin{equation}
A^{[l]} = \sigma \left( 
\begin{pmatrix}
w_{1,1}^{[l-1,l]} & w_{1,2}^{[l-1,l]} & \cdots & w_{1,m}^{[l-1,l]} \\
w_{2,1}^{[l-1,l]} & w_{2,2}^{[l-1,l]} & \cdots & w_{2,m}^{[l-1,l]} \\
\vdots & \vdots & \ddots & \vdots \\
w_{n,1}^{[l-1,l]} & w_{n,2}^{[l-1,l]} & \cdots & w_{n,m}^{[l-1,l]}
\end{pmatrix}
\begin{pmatrix}
A_1^{[l-1]} \\
A_2^{[l-1]} \\
\vdots \\
A_m^{[l-1]}
\end{pmatrix}
+
\begin{pmatrix}
b_1^{[l]} \\
b_2^{[l]} \\
\vdots \\
b_n^{[l]}
\end{pmatrix}
\right)
\end{equation}

\subsubsection{Schicht-für-Schicht-Propagation}
Beginnend mit der Eingabeschicht \( A^{[0]} \), die im Wesentlichen die Eingabedaten \( X \) sind, berechnet jede nachfolgende Schicht \( Z^{[l]} \) und \( A^{[l]} \) entsprechend den oben genannten Gleichungen \cite[p.~1421]{russell2021ai}.

\subsubsection{Dimensionalität und Netzwerkarchitektur}
Die Anzahl der Neuronen in jeder Schicht und die Art der verwendeten Aktivierungsfunktion können die Leistung des Netzwerks erheblich beeinflussen \cite[p.~1408]{russell2021ai}. Es ist wichtig, die Dimensionalität jeder Schicht während der Entwurfsphase zu berücksichtigen, um ein effektives Lernen sicherzustellen \cite[p.~73]{Chollet2021}.








%Informationstechnologie
  %Einleitung und Bedeutung der IT

%Grundlagen Neuronaler Netze
  %Einleitung und Historie
  %Nichtlineare Approximation
  %Architekturen neuronaler Netze (Feedforward, CNN, RNN)
  %Einführungsbeispiel: Funktionsapproximation
  %Fehlerberechnung und Kostenfunktion

%Optimierungsverfahren
  %Gradient Descent
  %Backpropagation
  %Advanced Optimizers (Adam, RMSprop)

%Hyperparameter und Regularisierung
  %Learning Rate
  %Batch Size
  %Overfitting und Regularisierung
  %Datenvorbereitung und -transformation
  %Normalization (Batch, Layer)
  
%Aktivierungsfunktionen
  %Sigmoid, ReLU, Tanh etc.

%Evaluationsmetriken
  %Klassifikation: Genauigkeit, F1-Score etc.
  %Reinforcement Learning: Reward-Metriken

%Reinforcement Learning
  %Grundlagen und Prinzipien
  %Markov-Entscheidungsprozesse (MDPs)
  %Belohnungssystem
  %Exploration vs Exploitation
  %Policy vs. Value-based Methods

%Regularisierung in RL
  %Entropie-Regularisierung etc.

%Fortgeschrittene Themen in Deep Learning
  %Deep Q-Learning
  %Actor-Critic-Methoden
  %DDPG (Deep Deterministic Policy Gradients)


%^
%^#### 3. Technische Umsetzung / Implementierung und Software-Architektur
%^- Programmiersprachen und Bibliotheken
%^  - C++
%^  - TensorFlow
%^  - Python
%^  - SystemC
%^- Schnittstellen-Design
%^  - API-Entwicklung
%^  - Modulare Architektur
%^- SystemC in der Schaltungssimulation
%^  - Grundlagen
%^  - Transientenanalyse
%^  - Vorteile und Herausforderungen
%^- Implementierungsdetails
%^  - Code-Architektur
%^  - Optimierung für Performanz
%^
%^---
%^
%^Die Bezeichnung "Technische Umsetzung" oder "Implementierung und Software-Architektur" für den dritten Teil würde in diesem Kontext sehr gut passen, da es sowohl den Programmcode als auch die spezifischen Software-Tools abdeckt, die für die Umsetzung der in den Grundlagen vorgestellten Konzepte erforderlich sind. So schaffen Sie eine klare Abgrenzung zu den vorherigen Teilen, die sich mehr mit den theoretischen Grundlagen befassen.













%Die Informationstechnologie (IT) hat eine tiefgreifende und weitreichende Wirkung auf fast jeden Aspekt der modernen Gesellschaft, einschließlich Wissenschaft, Industrie, Bildung und Alltagsleben. Die historische Entwicklung der IT ist ein faszinierendes Thema, das sich von den Anfängen der mechanischen Rechenmaschinen bis zu den heutigen Supercomputern und Cloud-Infrastrukturen erstreckt.
%
%### Frühe Anfänge
%Die Wurzeln der IT gehen auf mechanische Rechenmaschinen wie die Pascaline von Blaise Pascal und die Leibnizsche Rechenmaschine zurück. Diese Vorläufer der modernen Computer waren jedoch in ihren Fähigkeiten sehr eingeschränkt.
%
%### Das Zeitalter der Elektronischen Computer
%Mit dem Aufkommen der ersten elektronischen Computer in den 1940ern, wie dem ENIAC und dem UNIVAC, begann die moderne Ära der Informationstechnologie. Diese Maschinen verwendeten Vakuumröhren und waren oft raumgroß, aber sie legten den Grundstein für die Entwicklung kleinerer, leistungsfähigerer und effizienterer Systeme.
%
%### Miniaturisierung und Massenmarkt
%Der Übergang von Vakuumröhren zu Transistoren und später zu integrierten Schaltkreisen führte zur Miniaturisierung von Computern. Der Personal Computer (PC) revolutionierte in den 1980er Jahren die Zugänglichkeit von Computern für die breite Öffentlichkeit. 
%
%### Internet und Globalisierung
%Das Aufkommen des Internets in den 1990ern veränderte die Art und Weise, wie wir kommunizieren, Geschäfte tätigen und Informationen teilen. Es führte zu einer noch nie dagewesenen Globalisierung und Vernetzung.
%
%### Einfluss auf die Wissenschaft
%In der Wissenschaft hat die IT die Forschungsmethoden und -möglichkeiten erheblich erweitert. Supercomputer ermöglichen komplexe Simulationen in Bereichen wie Klimaforschung, Genomik und Materialwissenschaft. Datenanalyse-Tools haben das Handling von großen Datensätzen erleichtert, und künstliche Intelligenz hat Bereiche wie maschinelles Lernen und automatische Entscheidungsfindung vorangetrieben.
%
%### Einfluss auf die Industrie
%In der Industrie hat die IT Produktionsprozesse optimiert, die Lieferkette effizienter gemacht und den globalen Handel erleichtert. Automatisierung, Prozesskontrolle und Qualitätssicherung sind nur einige der Bereiche, die durch die IT revolutioniert wurden.
%
%Die fortschreitende Entwicklung in der Informationstechnologie zeigt keine Anzeichen einer Verlangsamung, mit fortlaufenden Innovationen in Bereichen wie Künstlicher Intelligenz, Internet der Dinge (IoT) und Quantencomputing. Diese Technologien versprechen, die Art und Weise, wie wir leben und arbeiten, weiter zu transformieren.
